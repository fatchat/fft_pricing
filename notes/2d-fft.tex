\documentclass[10pt]{amsart}
\usepackage{amssymb,amsmath, amsthm, amsfonts, listings}

\newcommand{\vect}[1]{\boldsymbol{#1}}
%\newcommand{\vect}[1]{\vec{#1}}

\newtheorem*{notation}{Notation}

\begin{document}

\section{Notation}
\begin{align}
 b &:= \text{number of components in the basket} \\
 \vect{X} &:= (X_0,\ldots,X_{b-1}) \\
 \vect{e} &:= (e_0,\ldots,e_{b-1}) \\
 N &:= \text{a positive integer} \\
 K &= \{ (k_0, \ldots, k_{b-1}) : k_j \in \{0, \ldots, N-1 \} \} \\
 \vect{k} &\in K \\
 \vect{I}(\alpha) &:= (\alpha, \ldots, \alpha) \qquad \text{($b$ times)} \\
 \vect{u}(\vect{k}) &:= \Delta (\vect{k} - \vect{I}(N/2)) + i\vect{e} \\
 \Phi &:= \text{characteristic function of spread/basket payoff} \\
 \widehat{P} &:= \text{Fourier transform of spread/basket payoff} \\
 F(\vect{k}) &= \Phi(\vect{u(k)}) \widehat{P}(\vect{u(k)}) \\
 \Delta &:= \text{some real number, maybe 1/4} \\
 \lambda &= \frac{2\pi}{N\Delta} \\
 \vect{X}(\vect{k}) &= \vect{X} + \lambda \vect{k} \\
 \vect{f} &= -\vect{e}\Delta - \vect{I} \left(i \frac{\Delta N}{2} \right) \\
 \vect{\ell} &\in K \\
 \vect{v}(\vect{\ell}) &= \vect{X} + \frac{2 \pi}{N \Delta} \vect{\ell} \\
 V(\vect{\ell}) &= \vect{f} \cdot \vect{v(\ell)}
\end{align}

\section{Idea}
Consider the following: whilst the function $\Phi$ is identical for both spread  and basket options,
we have
\begin{align}
 \widehat{P}& (\xi)=\frac{\prod_{j=1}^b\Gamma(-i\xi_j)}{\Gamma(-i(\sum_{j=1}^b\xi_j)+2)},
 \quad\text{ basket}\\
 \widehat{P}& (\xi)=\frac{\Gamma(i(\xi_1+\xi_2)-1)\Gamma(-i\xi_2)}{\Gamma(i\xi_1+1)}, \quad
 \text{ spread}\\
  D &:= \frac{e^{-rT}\Delta^b}{(2\pi)^b}
  \sum_{\vect{k} \in K}
  e^{i\left( \Delta (\vect{k} + i\vect{e}) \cdot \vect{X} - \Delta N \vect{X} / 2 \right)} F(\vect{k}) \\
  &= \frac{e^{-rT}\Delta^b}{(2\pi)^b}
  e^{V(\vect{\ell})}
  \sum_{\vect{k} \in K}
  e^{i \frac{2\pi}{N} (\vect{k} \cdot \vect{\ell})}
  e^{i\Delta (\vect{k} \cdot \vect{X})} F(\vect{k}) \\
  &= \frac{e^{-rT}\Delta^b}{(2\pi)^b}
  e^{V(\vect{\ell})}
  \sum_{\vect{k} \in K}
  e^{i \frac{2\pi}{N} (\vect{k} \cdot \vect{\ell})}
  A_{\vect{k}}
\end{align}
with
\begin{equation}
A_{\vect{k}} := e^{i \Delta (\vect{k} \cdot \vect{X})} F(\vect{k})
\end{equation}
\begin{equation}
A_{\vect{k}} \xrightarrow{FFT} B_{\vect{\ell}}
\end{equation}
and the call option price for the share values $\vect{X(\ell)}$ is
\begin{equation}
C_{\vect{\ell}} = \frac{e^{-rT} \Delta^b} {(2\pi)^b} e^{V(\vect{\ell})} B_{\vect{\ell}}
\end{equation}

\section{Implementation}

\begin{itemize}
\item The set $K$ is created by \lstinline!TupleFactory::generate_tuples_mod_N()!.
\item We have $\#K = N^b$.
\item All $b$-vectors are created using the LinAlgSys class.
\item Currently the entire implementation is in \lstinline!Basket::basketpricingFFT()!
\end {itemize}


\end{document}