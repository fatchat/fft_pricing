\documentclass[10pt]{amsart}
\usepackage{amssymb,amsmath, amsthm, amsfonts}

\newcommand{\rd}{\mathbb{R}^d}
\newcommand{\rtwo}{\mathbb{R}^2}
\newcommand{\rthree}{\mathbb{R}^3}
\newcommand{\R}{\mathbb{R}}
\newcommand{\N}{\mathbb{N}}
\newcommand{\zed}{\mathbb{Z}}
\newcommand{\pd}{\frac{d(d+1)}{d^2-d+2}}
\newcommand{\qd}{\frac{d+1}{d-1}}
\newcommand{\pq}{L^p\to L^q}
\newcommand{\pqd}{L^p(\mathbb{R}^d)\to L^q(\mathbb{R}^d)}
\newcommand{\ca}{\mathcal{A}}
\newcommand{\h}{\mathcal{H}}
\newcommand{\m}{\mathcal{M}}
\newcommand{\cc}{\mathcal{C}}
\newcommand{\cs}{\mathcal{S}}
\newcommand{\ct}{\mathcal{T}}
\newcommand{\qq}{\mathcal{Q}}
\newcommand{\si}{\mathfrak{S}}
\newcommand{\tef}{\langle X\chi_E,\chi_F\rangle}
\newcommand{\tab}{\langle X\chi_A,\chi_B\rangle}
\newcommand{\tfe}{\langle \chi_E,X^*\chi_F\rangle}
\newcommand{\epo}{\epsilon_1}
\newcommand{\ept}{\epsilon_2}
\newcommand{\eto}{\eta_1}
\newcommand{\ett}{\eta_2}

\newtheorem{thm}{Theorem}
\newtheorem{cor}{Corollary}
\newtheorem{lem}{Lemma}
\newtheorem{prop}{Proposition}
\newtheorem{defn}{Definition}
\newtheorem{rem}{Remark}
\newtheorem{rems}{Remarks}
\newtheorem*{notation}{Notation}

\begin{document}
Sorry for the previous mess - I did want to ask you whether I should start the formulas from 0 rather than from 1, and you already gave me the answer. Anyway, this is the multidimensional version, where the
dimension $b$ refers to the number of components in the basket. We therefore
define the vectors
\[X=(X_1,\ldots,X_b),\quad E=(\epsilon_1,\ldots,\epsilon_b),\quad K=(k_1,\ldots,k_b)\]
We wish to implement the formula
\begin{equation}\label{1}D\equiv \frac{e^{-rT}\Delta^b}{(2\pi)^b}\sum_{k_1,\ldots,k_b=0}^{N-1}
e^{i\left(\Delta(K+iE)\cdot X-\Delta N X/2\right)}F(K),
\end{equation}
where
\begin{multline*}F(K)= 
\Phi\left(\Delta(k_1-N/2)+i\epo,\ldots,\Delta(k_b-N/2)+i\epsilon_b\right)\\ \times \widehat{P}
\left(\Delta(k_1-N/2)+i\epo,\ldots,\Delta(k_b-N/2)+i\epsilon_b\right),%\]
\end{multline*}
$\Phi$ is the characteristic function and $\widehat{P}$ is the Fourier transform of the payoff (and this can be the payoff for spread options or basket options).
Now by setting \[ X_j=X_{0j}+\lambda\ell_j,\quad j=1,\ldots,b\quad\ell_j=0,\ldots,N-1\]
and \[\lambda=\frac{2\pi}{N\Delta},\quad L=(\ell_1,\ldots,\ell_b)\]
 we rewrite (\ref{1}) as 
\begin{equation}\label{2}\frac{e^{-rT}\Delta^b}{(2\pi)^b} e^{\sum_{j=1}^bV_j(\ell_j)}
\sum_{k_1,\ldots,k_b=0}^{N-1}
e^{i\frac{2\pi}{N}K\cdot L}e^{i\Delta(\sum_{j=1}^b k_jX_{0j})}F(K),
\end{equation}
with
\[V_j(\ell_j)=\left(-\epsilon_j-i\Delta N/2\right)\left(X_{0j}+\frac{2\pi}{N\Delta}\ell_j\right),\quad j=1,\ldots,b.\] 
We now have the \textbf{new observation} that for each $j$ we may write
\[V_j(\ell_j)=-\epsilon_j\left(X_{0j}+\frac{2\pi}{N\Delta}\ell_j\right)-i\Delta N X_{0j}/2-i\pi\ell_j,\]
so that
\[e^{\sum_{j=1}^bV_j(\ell_j)}=\left(\prod_{j=1}^b\left(-1\right)^{\ell_j}\right)e^{-\sum_{j=1}^b \epsilon_j \left(X_{0j}+\frac{2\pi}{N\Delta}\ell_j\right)
-i\sum_{j=1}^b\Delta N X_{0j}/2}.\]
Note that inside the summation we have the b-dimensional discrete Fourier transform of the 
tensor \[A_{K}\equiv e^{i\Delta(\sum_{j=1}^b k_jX_{0j})}F(K);\]
its Fast Fourier transform will produce a tensor \[B_{L}=\{b_{\ell_1,\ldots \ell_b}\};\] 
now the tensor
\[C=\{c_{L}\}=\frac{e^{-rT}\Delta^b}{(2\pi)^b} e^{\sum_{j=1}^bV_j(\ell_j)}
b_{\ell_1,\ldots,\ell_b} \]
will give the call option price for each choice of share values
\[X_{\ell_j}=(X_{0j}+\lambda\ell_j),\quad j=1,\ldots,b\quad\ell_j=0,\ldots,N-1.\]
Hopefully this is clear now - by the way, as I mentioned I think our definition of Discrete Fourier
transform is the backwards FFT in the Fastest Fourier Transform in the West.


























\end{document}